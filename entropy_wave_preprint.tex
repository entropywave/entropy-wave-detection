
\documentclass[12pt]{article}
\usepackage{amsmath}
\usepackage{geometry}
\usepackage{authblk}
\usepackage{setspace}
\usepackage{hyperref}
\geometry{margin=1in}
\setstretch{1.2}

\title{Global Phase Coherence Drift as a Probe of Cosmological Entropy Fluctuations}
\author{Steve Pence}
\affil{Entropy Wave Detection Initiative\\Spring 2025\\\texttt{entropywaves@protonmail.com}}
\date{}

\begin{document}

\maketitle

\begin{abstract}
We propose a novel experimental framework to detect cosmological-scale entropy fluctuations through passive monitoring of idle-mode quantum coherence. By tracking long-duration phase drift and decoherence in superconducting qubit systems across multiple geographically separated laboratories, and applying statistically rigorous cross-correlation and environmental noise rejection methods, we aim to identify entropy-relevant global informational instabilities. This work defines a new observational domain complementary to gravitational wave detectors by focusing on coherence-level informational distortions rather than classical spacetime strain. The proposed approach leverages existing quantum computing infrastructure and is designed for 5$\sigma$ significance detection of correlated anomalies.
\end{abstract}

\section*{1. Introduction}
The last decade has seen quantum computing platforms achieve sufficient stability to allow precise characterization of decoherence processes in isolated quantum systems. These advances open the possibility for their use not only in computation, but as precision sensors of fundamental phenomena.

This work introduces the hypothesis that entropy fluctuations---changes in the effective number of accessible microstates of a system, whether due to vacuum field behavior, topological shifts in Hilbert space, or cosmological-scale informational distortions---may be indirectly measurable through correlated coherence anomalies in qubit systems.

We present a detection architecture that is fully passive, uses existing qubit coherence monitoring protocols (e.g. Ramsey interferometry), and relies on statistical correlation of globally observed phase drift under strict noise rejection. The proposal includes a detailed environmental and instrumental noise rejection protocol, and defines criteria for 5$\sigma$ confidence level in event validation.

\section*{2. Motivation}
While LIGO and similar observatories detect spacetime strain from astrophysical events, no comparable instrument exists to detect disturbances in the information-carrying structure of the quantum vacuum. We hypothesize that such disturbances may not produce mechanical effects, but may instead induce detectable phase drift or decoherence in quantum systems. 

This framework draws conceptual parallels with theories of vacuum entropy, dark sector coupling, and informational symmetry breaking. The proposed experiment may also serve to constrain or detect low-frequency, non-gravitational vacuum perturbations consistent with entropy wave propagation.

\section*{3. Experimental Architecture (Summary)}
\begin{itemize}
  \item Superconducting qubits prepared in idle-mode superposition states
  \item GPS-synchronized decoherence and phase drift monitoring
  \item Environmental sensor arrays (EM, seismic, thermal)
  \item Centralized correlation engine for cross-lab anomaly detection
  \item Monte Carlo-based null model simulation for background rejection
  \item Blind injection capability for false positive resistance
\end{itemize}

\section*{4. Intellectual Contribution}
This project and methodology were conceived and developed by Steve Pence. All protocols, analysis frameworks, and scientific motivation are original to this work. Collaboration with quantum laboratories is sought for implementation under joint authorship.

\section*{5. Status and Next Steps}
Documentation is complete and available. Outreach to experimental labs is in progress. We are seeking co-investigators, experimental hosts, and funding support for Phase 1 calibration and Phase 2 long-duration observation.

\end{document}
